\documentclass{article}

\begin{document}
\pagestyle{plain}

\begin{center}
{\LARGE \bf Synphot Data Installation Guide}
\end{center}

\vspace{6ex}

\section{Setting the Top Directory}

The synphot tasks assume that all the synphot reference files are
stored under a single top level directory. This directory is referred
to inside STSDAS by the logical name {\tt crrefer}. This directory
may be anywhere you have sufficient space to install the reference
files (approximately 400 megabytes is required for the full
installation), but we recommend that it not be placed as subdirectory
of the STSDAS or TABLES source code. This will make it easier to
update STSDAS without needing to reinstall the Synphot data. Once the
top directory is created, the environment variable {\tt crrefer}
should be set in your {\tt extern.pkg} file. This file can be found in
{\tt \$iraf/unix/hlib} on Unix systems or {\tt \$iraf/vms/hlib} on VMS
systems. To set {\tt crrefer} add a command similar to the following
to the file:

\begin{quote}\begin{verbatim}
set crrefer = "/your/path/name/to/refer/"
\end{verbatim}\end{quote}

The trailing slash is important, so do not omit it. VMS users should
use a VMS directory name in place of the Unix directory name shown
above.

\section{Unpacking the Data Sets}

The Synphot data should be downloaded from our anonymous ftp at:

\begin{quote}\begin{verbatim}
http://www.stsci.edu/ftp/software/stsdas/refdata/synphot/ 
\end{verbatim}\end{quote}

If you do not have access to anonymous ftp, you can contact our STSDAS
system administrator, {\tt colbert@stsci.edu}, and have him make a
tape containing the necessary files. There are four compressed tar
files containing the data and this installation guide. The first tar
file contains the Synphot component throughput tables, the second
contains various observed and modelled spectral coatalogs, the
third contains the 1993 Kurucz model stellar spectra, and the fourth
contains the HST calibration standard spectra.

First, place the compressed tar files in the top level directory you
created in the first section. Then, uncompress and untar the tar
files. On a Unix system, the following commands will accomplish this.

\begin{quote}\begin{verbatim}

% uncompress synphot1.tar.Z 
% tar -xvf synphot1.tar

% uncompress synphot2.tar.Z 
% tar -xvf synphot2.tar

% uncompress synphot3.tar.Z 
% tar -xvf synphot3.tar

% uncompress synphot4.tar.Z 
% tar -xvf synphot4.tar

\end{verbatim}\end{quote}

On a VMS system, the equivalent commands are most easily executed in
the IRAF environment. If you do not have a VMS version of the
uncompress program, it can be downloaded from our anonymous ftp at

\begin{quote}\begin{verbatim}
http://www.stsci.edu/ftp/software/stsdas/util/
\end{verbatim}\end{quote}

\begin{quote}\begin{verbatim}

da> !uncompress synphot1.tar_z
da> rtar -xtvfn synphot1.tar

da> !uncompress synphot2.tar_z
da> rtar -xvf synphot2.tar    

da> !uncompress synphot3.tar_z
da> rtar -xvf synphot3.tar    

da> !uncompress synphot4.tar_z
da> rtar -xvf synphot4.tar    

\end{verbatim}\end{quote}

The tar file synphotpsf.tar.Z contains the psf images used with the
simulators package of synphot. If you are not planning to use this
package, you do not need to install it. The tar file should be copied
to the {\tt scidata} directory of stsdas, uncompressed, and untarred.


Type the following commands when in stsdas:

\begin{quote}\begin{verbatim}

cl> copy /your/path/to/synphotpsf.tar.Z scidata$
cl> cd scidata$
cl> !uncompress synphotpsf.tar.Z
cl> rtar -xvf synphotpsf.tar

\end{verbatim}\end{quote}

\section{Updating the Synphot Parameter File}

We have converted all the Synphot files to FITS format. For the most
part, this change is transparent to you, but in order to have Synphot
use the new files, you must edit the {\tt refdata} parameter set. The
new parameter set should have the values:

\begin{quote}\begin{verbatim}
area = 45238.93416 (unchanged)
grtbl = mtab$*_tmg.fits
cmptbl = mtab$*_tmc.fits
\end{verbatim}\end{quote}

Each user can edit the parameter set with the command {\tt epar
refdata} in Iraf. Or, to make the change for all users on the system,
modify the file {\tt synphot\$refdata.par} with a text editor. (The
name is an iraf logical. You can see the name on your system by
loading the {\tt synphot} and {\tt hst\_calib} packages in iraf and
typing ``{\tt path synphot\$refdata.par}''.)

\section{When Disaster Strikes}

If you encounter problems installing the Synphot data files, we
encourage you to contact us via the STSDAS help desk {\tt
help@stsci.edu}.  If you are unable to use this mechanism, please call
the help desk at 1-800-544-8125 or (410) 338--1082 or write us at

\begin{quote}
Scientific Software Group\\
Science Support Division\\
Space Telescope Science Institute\\
3700 San Martin Drive\\
Baltimore, MD 21218
\end{quote}

If you have any suggestions for improvement in either the installation
procedure or this gude, please feel free to contact us.

\end{document}
